\chapter{Theoretische Grundlagen}
\label{ch:grundlagen}

\section{Grundlagen und Verfahren der Luftzerlegung}
\label{sec:grundlagen_luftzerlegung}

Luft weist weltweit, abgesehen von geringen Verunreinigungen, eine nahezu identische Zusammensetzung auf (vgl. \ref{tab:a_Luftzusammensetzung}). Die Hauptbestandteile Stickstoff, Sauerstoff und Argon sind sowohl in flüssiger als auch in gasförmiger Form von großer industrieller Bedeutung, des Weiteren finden weitere Edelgase wie Neon, Krypton und Xenon vielfältige Anwendungen. \\
Stickstoff ist inert und nicht toxisch, mit zahlreichen Verwendungszwecken entfällt der größte Anteil auf seine Nutzung als Rohstoff für die Ammoniaksynthese im Haber-Bosch-Verfahren. Die globale Ammoniakproduktion zeigt in den letzten Jahren einen stetig steigenden Trend von rund \SI{180}{\mega\tonne} pro Jahr im Jahr 2017, was etwa \SI{150}{\mega\tonne} Stickstoff entspricht \cite{mingolla_low-carbon_2025,MediumTermFertilizer}, bis zu rund \SI{190}{\mega\tonne} im Jahr 2024, entsprechend etwa \SI{155}{\mega\tonne} bereitgestelltem Stickstoff \cite{IFAShortTermFertilizerOutlook2025}.
Marktanalysen beziffern das Volumen für flüssigen Stickstoff (ohne Ammoniaksynthese) im Jahr 2024 auf etwa \SI{125.64}{\mega\tonne} \cite{noauthor_liquid_2025,kiran_pulidindi_marktgrose_2024} und für gasförmigen Stickstoff auf rund 1{,}5\,Mio.\,t \cite{noauthor_industrial_2025,pulidindi_nitrogen_2024}. Prognosen bis 2030 gehen von einem weiteren Anstieg aus. Stickstoff wird vor allem als Inert- und Schutzgas in der chemischen Industrie, der Metallverarbeitung sowie in der Elektronik- und Halbleiterfertigung eingesetzt. Weitere zentrale Anwendungen liegen in der Lebensmittelindustrie (Kühlung, Verpackung), in der Medizin und Biotechnologie (Kryotherapie, Konservierung biologischer Proben) sowie in der Energie- und Ölindustrie (Spülen, Inertisierung) \cite{nick_bocker_ullmanns_2000, noauthor_liquid_2025}. \\
Für Sauerstoff wird das weltweite Marktvolumen auf rund \SI{562.8}{\mega\tonne} geschätzt.  Reiner Sauerstoff wird vor allem in der Stahlerzeugung (ca. \SI{55}{\percent} der Produktion) sowie in chemischen Prozessen als Oxidationsmittel (ca. \SI{25}{\percent}) eingesetzt. Weitere Anwendungen finden sich in der Wasser- und Abwasseraufbereitung, der Metallverarbeitung, der Raumfahrt, der Medizin und Biotechnologie sowie in der Lebensmittelindustrie. \cite{nick_bocker_ullmanns_2000, heinz-wolfgang_haring_air_2007}. Zunehmende Bedeutung gewinnt Sauerstoff im Oxyfuel-Verfahren, das durch die Verbrennung ohne Luftstickstoff eine effiziente CO₂-Abscheidung (Carbon Capture) in der Zement- und Glasindustrie ermöglicht. Diese Technologie gilt als Schlüssel zur Dekarbonisierung, da sie hochkonzentrierte CO₂-Abgasströme zur Speicherung oder Weiterverwendung liefert \cite{raho_technological_2025} \\
Argon wird dort eingesetzt, wo Stickstoff als Inertgas ungeeignet ist, etwa beim Schweißen von Metallen wie Titan, Tantal oder Wolfram. Darüber hinaus dient es als Löschmittel, als Verpackungsgas in der Lebensmittelindustrie, als Füllgas für Glühbirnen, als Trägergas in der Gaschromatographie sowie als Inert- und Schneidgas in der Lasertechnik. \cite{nick_bocker_ullmanns_2000, noauthor_industrial_2025, heinz-wolfgang_haring_air_2007} \\
\\
Im Allgemeinen stehen drei kommerziell verfügbare Verfahren zur Luftzerlegung zur Verfügung: Membrantrennung, Adsorptionsanlagen sowie die kryogene Destillation. \\
Die nicht-kryogenen Luftzerlegungsverfahren, wie Membrananlagen und die Druckwechseladsorption (Pressure Swing Adsorption, \gls{PSA}) beziehungsweise die Vakuumwechseladsorption (Vacuum Swing Adsorption, \gls{VSA}) sind vor allem für kleinere und mittlere Gasverbräuche relevant. Sie liefern typischerweise Ströme mit mittlerer Reinheit (z.\,B.\ Stickstoff 98--99,5\,Vol.-\% bei der Druckwechseladsorption, Sauerstoff bis etwa 95\,Vol.-\% bei der Vakuumwechseladsorption). Während Membrananlagen in der Regel für kleine Durchsätze (häufig im Bereich von unter \SI{10}{\cubic\metre\per\hour}) wirtschaftlich betrieben werden, liegt der typische Einsatzbereich von Adsorptionsanlagen im Bereich von etwa \SI{10}{\cubic\metre\per\hour} bis \SI{100}{\cubic\metre\per\hour} \cite{heinz-wolfgang_haring_air_2007, cornelissen_exergy_1998}. Ein wesentlicher Vorteil dieser Verfahren liegt in ihrer hohen Flexibilität. Sie können innerhalb weniger Minuten gestartet und auf volle Leistung hochgefahren werden, damit lassen sich stark schwankende Bedarfe effizient abdecken. \cite{heinz-wolfgang_haring_air_2007, cornelissen_exergy_1998}\\
Die kryogene Destillation ist das weltweit am häufigsten eingesetzte Verfahren zur Luftzerlegung und kommt immer dann zum Einsatz, wenn besonders hohe Reinheiten, große Mengen, flüssige Produkte oder auch Argon benötigt werden. Für 2024 wird berichtet, dass ein Großteil der weltweit betriebenen Luftzerlegungsanlagen auf kryogenen Verfahren basiert (in Marktübersichten teils mit Größenordnungen um \SI{75}{\percent} angegeben); dabei stellt die Stickstoffproduktion mit einem Anteil von etwa \SI{37.6}{\percent} den größten Einzelmarkt dar \cite{noauthor_air_2025, ankit_gupta_air_nodate}. Prognosen gehen zudem von einem anhaltenden jährlichen Wachstum des Marktes für Luftzerlegungsanlagen aus, das insbesondere durch die steigende Nachfrage im asiatisch-pazifischen Raum getragen wird. Der Grund für die Dominanz der kryogenen Destillation liegt in den deutlich geringeren spezifischen Kosten im großtechnischen Maßstab sowie in der Möglichkeit, sehr große Mengen bei gleichzeitig hoher Reinheit der erzeugten Gase bereitzustellen.
Die Produkte einer Luftzerlegungsanlage werden entweder gasförmig über Pipelines oder in verflüssigter Form per Lkw transportiert. Gasförmige Produkte werden vor allem in Pipelinesysteme zu Stahlwerken oder integrierten Vergasungskombikraftwerken innerhalb des Industriegebiets eingespeist, der Transport von Flüssigprodukten erfolgt in Gasflaschen, Lagertanks oder Tankwagen. \cite{linde_engineering_air_2019} \\
In den vergangenen Jahrzehnten konnten erhebliche technologische Verbesserungen in der kryogenen Luftzerlegung erzielt werden, die sowohl die Produktivität der kryogenen Destillationsanlagen als auch ihre Energieeffizienz gesteigert haben. Dennoch liegt der spezifische Energiebedarf nach wie vor über der theoretisch erforderlichen Trennungsenergie, sodass auch künftig weitere Effizienzsteigerungen zu erwarten sind \cite{darde_air_2009}. Die kryogene Destillation ist somit nicht nur das dominierende Verfahren der Luftzerlegung, sondern bietet zugleich noch erhebliche Optimierungsmöglichkeiten. Vor diesem Hintergrund konzentriert sich die vorliegende Arbeit auf diese Technologie.\\

\section{Historische Entwicklung und Grundlagen der kryogenen Luftzerlegung}
\label{sec:grundlagen_historie_kryogene}

Die Entwicklung der kryogenen Luftzerlegung baut auf der grundlegenden Entdeckung der Luftverflüssigung aus dem 19.~Jahrhundert auf. Joule und Thomson beschrieben 1852, dass reale Gase bei der Expansion durch ein Drosselventil abkühlen (Joule-Thomson-Effekt). Van der Waals lieferte 1873 die theoretische Grundlage, indem er intermolekulare Kräfte in sein Gasmodell einbezog. Damit wurde deutlich, dass sich Gase durch wiederholte Verdichtung und Expansion schrittweise so weit abkühlen lassen, dass eine technische Verflüssigung möglich wird. Erste Experimente zur Luftverflüssigung durch Cailletet und Pictet im Jahr 1877 führten zwar zur Bildung kleiner Flüssigkeitströpfchen, jedoch noch nicht zu einem kontinuierlichen oder industriell nutzbaren Verfahren. Carl von Linde gelang 1895 in München der entscheidende Durchbruch zum ersten kontinuierlichen Verfahren der Luftverflüssigung, das auf dem Joule-Thomson-Effekt in Kombination mit einem Gegenstrom-Wärmeaustauscher basierte. Dabei wurde komprimierte Luft schrittweise durch Rückführung mit bereits abgekühlten Gasströmen weitergekühlt und schließlich durch Expansion verflüssigt. Mit diesem Verfahren konnte erstmals eine kontinuierliche Produktion von Flüssigluft erreicht werden, was den Beginn der industriellen Tieftemperaturtechnik markierte. \cite{hausen_tieftemperaturtechnik_1985, linde_engineering_air_2019}\\
\\
Zu Beginn des 20.~Jahrhunderts war mit der Grundlage der Luftverflüssigung die Voraussetzung geschaffen, Luftbestandteile großtechnisch voneinander zu trennen. Carl von Linde errichtete 1902 die erste Luftzerlegungsanlage, in der flüssige Luft mittels Rektifikation in ihre Hauptkomponenten Sauerstoff und Stickstoff getrennt wurde. In diesem frühen Einsäulenprozess ließ sich bereits reiner Sauerstoff gewinnen, während der Stickstoffstrom noch einen Restgehalt von etwa sieben Prozent Sauerstoff enthielt \cite{hausen_tieftemperaturtechnik_1985}. Ein entscheidender Fortschritt Lindes erfolgte 1910 mit der Einführung des Zweisäulenverfahrens. Dabei wurden eine Hochdruck- und eine Niederdruckkolonne thermisch gekoppelt, sodass erstmals hochreiner Sauerstoff und Stickstoff gleichzeitig produziert werden konnten. Dieses Prinzip der Doppelsäule mit einem Kondensator am Kopf der Hochdruckkolonne, der gleichzeitig als Verdampfer für die Niederdruckkolonne dient, ist bis heute die Grundlage moderner Luftzerlegungsanlagen. \cite{hausen_tieftemperaturtechnik_1985, linde_engineering_air_2019} \\
\\
Heutige kryogene Luftzerlegungsanlagen werden je nach Produktfraktionen (O\textsubscript{2}, N\textsubscript{2}, Ar), geforderter Reinheit und Durchsatz in zahlreichen Konfigurationen realisiert. Der Aufbau lässt sich grundsätzlich in die folgenden Funktionsblöcke gliedern:
\begin{itemize}
    \item Luftverdichtung 
    \item Luftreinigung
    \item (Haupt-)wärmeübertrager
    \item Rektifikationskolonne
    \item interne oder externe Produktverdichtung
\end{itemize}\\

Zunächst wird die Umgebungsluft nach einem mechanischen Vorfilter zur Staubentfernung in einem mehrstufigen Verdichter mit Zwischenkühlung auf den für die nachfolgende Verflüssigung und den Kolonnenbetrieb erforderlichen Zulaufdruck gebracht. Für klassische Doppelkolonnenprozesse wird dabei typischerweise ein Speisedruckniveau im Bereich von etwa 5--6\,bar eingesetzt \cite{linde_engineering_air_2019}. In der anschließenden Luftreinigung werden alle Komponenten entfernt, die bei kryogenen Temperaturen ausfrieren würden, insbesondere Wasserdampf und Kohlendioxid. \\
Im (Haupt-)Wärmetauscher wird die gereinigte Luft im Gegenstrom mit kalten Rückströmen bis nahe an die Verflüssigungstemperaturen abgekühlt. \\
Die eigentliche Trennung erfolgt in der Rektifikationssektion, im Ein- oder Zweisäulensystem. Für die Argonproduktion wird eine zusätzliche Kolonne nachgeschaltet. \\
Die Produktausführung (flüssig/gasförmig) und der Bereitstellungsdruck werden am Ende festgelegt, entweder durch externe oder durch interne Kompression. Bei externer Kompression werden die gasförmigen Produkte nach der Rektifikation im Hauptwärmetauscher auf Ziel- oder Umgebungstemperatur gebracht und anschließend mit Gaskompressoren auf den Verbraucherdruck verdichtet. Bei interner Kompression werden die Produkte als Flüssigkeit im isolierten Kaltbereich der Anlage (Coldbox) auf Druck gepumpt und im Wärmetauscher verdampft, sodass die Produkte (O\textsubscript{2}, N\textsubscript{2} oder Ar) direkt auf Zieldruck bereitgestellt werden können. \cite{tesch_comparative_2020, heinz-wolfgang_haring_air_2007}

\section{Luftzerlegung im Kontext der Ammoniaksynthese}
\label{sec:grundlagen_ammoinaksynthese}

Die Ammoniaksynthese nach Haber-Bosch basiert auf der folgenden Reaktionsgleichung:
\[
\ce{N2 + 3 H2 <=> 2 NH3} \qquad \Delta H = \SI{-92.4}{\kilo\joule\per\mol}
\]
Die Reaktion ist stark exotherm, das chemische Gleichgewicht verschiebt sich temperaturabhängig zugunsten des Ammoniaks bei niedrigen Temperaturen, während hohe Temperaturen die Reaktionsgeschwindigkeit fördern. Daher wird der Prozess bei hohen Drücken betrieben, um trotz der notwendigen erhöhten Reaktionstemperaturen eine ausreichende Gleichgewichtsausbeute zu erzielen. Die Edukte Stickstoff und Wasserstoff werden zuerst gemischt und erst danach gemeinsam verdichtet, da die Kompression des fertigen Gasgemisches energetisch effizienter und technisch einfacher zu steuern ist als die getrennte Verdichtung der Einzelgase. Der sogenannte Loop-Druck, der Betriebsdruck des Ammoniaksynthesekreislaufs, liegt zwischen \SI{150}{\bar} und \SI{300}{\bar} \cite{wiley-vch_ammonia_2011-1, wiley-vch_ammonia_2011-2, humphreys_development_2021}. Ältere Anlagen fahren teilweise bei noch höheren Drücken. Der Druck ist unter anderem vom eingesetzten Katalysator abhängig, typischerweise einem Eisen- oder Ruthenium-Katalysator. \cite{hansen_importance_2025, wiley-vch_ammonia_2011-1}\\
Die Reaktion läuft in der Regel bei Temperaturen zwischen \SI{400}{\celsius} und \SI{500}{\celsius} \cite{wiley-vch_ammonia_2011-1} \\

Die Reinheitsanforderungen an den Stickstoff für die Ammoniaksynthese sind besonders streng, da selbst Spuren von Verunreinigungen die Aktivität und Stabilität der eingesetzten Eisen- oder Rutheniumkatalysatoren stark beeinträchtigen können. Eine aktuelle Studie von Hansen et al. (2025) \cite{hansen_importance_2025} hat die Bedeutung der Stickstoffreinheit für den Haber-Bosch-Prozess umfassend quantifiziert.\\
Insbesondere Sauerstoff (\ce{O2}) muss auf wenige ppm (typisch $\leq$ 10 ppm) reduziert werden, da er die aktiven Zentren der Katalysatoren oxidiert und dadurch irreversibel deaktiviert. Hansen et al. \cite{hansen_importance_2025} zeigen, dass bereits \SI{5}{ppm} Sauerstoff im Stickstoffstrom zu einer messbaren Verringerung der Ammoniakausbeute führen. Auch Wasserdampf ist kritisch, denn es sind Taupunkte von \SIrange{-76}{-84}{^\circ C} gefordert, was nur wenigen ppm \ce{H2O} entspricht. Bereits geringe Mengen führen zur Oxidation der aktiven Oberflächen und zur Bildung stabiler Hydroxide, wodurch die Katalysatoraktivität stark herabgesetzt wird. Zudem begünstigt Wasserdampf Nebenreaktionen wie die Methanbildung im Syntheseloop. Kohlenstoffverbindungen wie Kohlenmonoxid (\ce{CO}) und Kohlendioxid (\ce{CO2}) gelten ebenfalls als starke Katalysatorgifte. \ce{CO} bindet mit hoher Affinität an die aktiven Zentren des Katalysators und blockiert diese dauerhaft, während \ce{CO2} unter den Reaktionsbedingungen teilweise zu \ce{CO} und atomarem Sauerstoff zerfällt und somit indirekt dieselbe giftige Wirkung entfaltet. Daher werden beide Komponenten in der Gasaufbereitung des Wasserstoffstroms vollständig entfernt. Schwefelverbindungen wie Schwefelwasserstoff (\ce{H2S}), Carbonylsulfid (\ce{COS}) oder Schwefeldioxid (\ce{SO2}) bilden schon in Spuren ($<$ \SI{0.1}{ppm}) stabile Metall-Sulfide auf den Katalysatoren und reduzieren deren Lebensdauer drastisch. Neben diesen reaktiven Komponenten stellt auch Argon ein Problem dar. Als inertes Gas akkumuliert es im Kreislauf der Synthese und muss über Purge-Ströme entfernt werden, was mit Verlusten an \ce{H2} und \ce{NH3} verbunden ist. Hansen et al.\ \cite{hansen_importance_2025} zeigen, dass eine Erhöhung des Argongehalts im Stickstoffstrom von \SI{0.5}{\percent} auf \SI{1.0}{\percent} den Purge-Strom um über \SI{30}{\percent} vergrößert und den spezifischen Energieverbrauch der Ammoniaksynthese um rund \SI{2}{\percent} erhöht. \cite{wiley-vch_ammonia_2011-2, humphreys_development_2021} \\
Insgesamt erfordert die Ammoniaksynthese einen Stickstoffstrom mit einer Reinheit von mindestens \SI{99.9}{\percent}, entsprechend der Qualitätsstufe \emph{N2~3.0} („high-purity nitrogen“) nach international gebräuchlicher Klassifikation (IUPAC; vgl. ISO~14175 \cite{IUPAC_GoldBook,ISO14175}). Dabei sind einzelne Verunreinigungen nur im ppm-Bereich tolerierbar und je höher die Stickstoffreinheit ist, desto stabiler und energieeffizienter läuft der Haber-Bosch-Prozess \cite{hansen_importance_2025, humphreys_development_2021}.\\
\\
Die Stickstoffbereitstellungsart und Konfiguration der eingesetzten Luftzerlegungsanlage hängt unmittelbar von der Wasserstoffbereitstellung bzw. Art des Ammoniaksyntheseverfahrens ab. Üblicherweise wird die Ammoniaksynthese nach einem Farbkonzept differenziert, das den zugrunde liegenden Primärenergieträger und die Klimabilanz der Ammoniakproduktion widerspiegelt.\\
Bei der klassischen grey ammonia („grauer Ammoniak“) wird Wasserstoff aus Dampfreformierung (\gls{SMR}) mit einem Luft-Sekundärreformer gewonnen. In dieser Kombination gelangt der Stickstoff direkt mit der eingesetzten Luft in den Prozessgasstrom. Dadurch entfällt die Notwendigkeit einer separaten Luftzerlegungsanlage für Stickstoff. Dieses Konzept stellt eine technisch elegante Lösung dar, da es zwei Funktionen gleichzeitig erfüllt: Der Sauerstoffanteil der Luft liefert die notwendige Reaktionswärme für die endotherme Reformierung, während der Stickstoff automatisch in den Syntheseloop eingebracht wird. Ein Nachteil besteht jedoch darin, dass auch Argon und andere inerte Luftbestandteile in den Kreislauf gelangen. Diese müssen über Purge-Ströme wieder ausgeschleust werden, was mit zusätzlichen Verlusten an Wasserstoff und Ammoniak verbunden ist. Aus diesem Grund ist diese Verfahrensweise heute vor allem in älteren und kostensensitiven Anlagen verbreitet, während moderne Großanlagen zunehmend andere Konzepte nutzen \cite{wiley-vch_ammonia_2011-1,smith_review_2001}.\\
Blue ammonia basiert ebenfalls auf Wasserstoff aus fossilen Rohstoffen, unterscheidet sich jedoch durch die Kombination mit Kohlendioxid-Abscheidung und -Speicherung (\gls{CCS}). In auto-thermischen oder partiell-oxidativen Reformierungsprozessen (\gls{ATR}/\gls{POX}) wird dabei reiner Sauerstoff benötigt, um eine kontrollierte Oxidation ohne Stickstoff- und Argonballast zu ermöglichen. Hier übernimmt die Luftzerlegungsanlage eine Doppelfunktion, da sie sowohl Sauerstoff für den Reformer als auch Stickstoff für die Ammoniaksynthese bereitstellt. Dadurch kann das stöchiometrische Verhältnis von \ce{N2} zu \ce{H2} präzise eingestellt werden, was die Energieeffizienz des Gesamtprozesses verbessert. Die Kopplung von \gls{ATR}/\gls{POX} mit einer etablierten Doppelkolonne als Luftzerlegungsanlage gilt daher heute als Standard für große, moderne blue ammonia-Anlagen. Der Vorteil liegt in der deutlichen Reduktion der Treibhausgasemissionen (bis zu \SI{90}{\percent}), dem steht jedoch ein erhöhter Energiebedarf für Sauerstoffbereitstellung und Kohlendioxid-Abscheidung gegenüber \cite{wiley-vch_ammonia_2011-1, humphreys_development_2021, smith_review_2001}.\\
Green ammonia gilt als klimaneutrale Zieltechnologie und basiert auf Wasserstoff aus Wasserelektrolyse mit erneuerbarem Strom. Hier wird die Luftzerlegungsanlage in der Regel als reine Stickstoffanlage betrieben, während Sauerstoff als Nebenprodukt der Elektrolyse anfällt und gegebenenfalls anderweitig genutzt oder vermarktet wird \cite{wiley-vch_ammonia_2011-1, humphreys_development_2021}. Für diese Stickstoff-only-Designs existieren mehrere technische Ansätze, sowohl kompakte Single-Column-Konzepte als auch Doppelkolonnen-Stickstoff-Generatoren, auf welche im Kapitel \ref{sec:grundlagen_forschungsstand} tiefgehender eingegangen wird. \\
Turquoise ammonia („türkiser Ammoniak“) aus Methanpyrolyse sowie brown/black ammonia („brauner/schwarzer Ammoniak“) auf Basis von Kohlevergasung werden in der Literatur ebenfalls genannt, spielen jedoch für den industriellen Stand der Technik derzeit eine untergeordnete Rolle.\\
\\
\textcolor{blue}{[aktuelle Green Ammonium Plants weltweit mit Info über N₂-Bereitstellung???? (hier aktuelle Beispiele?)]}

\section{Aktueller Stand von Forschung und Technik}
\label{sec:grundlagen_forschungsstand}

In der modernen Luftzerlegung kommen je nach Leistungsanforderung und geforderter Produktreinheit verschiedene Prozesskonfigurationen zum Einsatz. Der Doppelsäulenprozess bildet dabei den industriellen Standard. Während der ursprünglich entwickelte Einsäulenprozess aufgrund seines Potenzials zur Vereinfachung, Kostensenkung und besseren Skalierbarkeit erneut Gegenstand aktueller Forschung ist, werden zugleich neuartige Dreisäulenprozesse hinsichtlich weiterer Effizienzsteigerungen und verbesserter Argonrückgewinnung untersucht \parencite{ye_feasibility_2019}.

\subsection{Klassischer Doppelsäulenprozess}
\label{sec:grundlagen_doppelsäule}

Der von Carl von Linde entwickelte Doppelsäulenprozess (Double-Column Process) ist die technologisch ausgereifteste und am weitesten verbreitete Standardkonfiguration für die kryogene Luftzerlegung im industriellen Maßstab \cite{linde_engineering_air_2019, smith_review_2001, cornelissen_exergy_1998}. Die Verschaltung der Prozessströme durch die bereits benannten Funktionsblöcke (vgl. \ref{sec:grundlagen_historie_kryogene}), auf welche im Folgenden genauer eingegangen werden soll, ist in Abbildung \ref{fig:luftzerlegung_intern} dargestellt. 

Im Luftverdichtungsblock wird die Umgebungsluft in meist mehrstufigen Turboverdichtern mit Zwischenkühlung (Wasser- oder Luftkühlung) auf den für den Kolonnenbetrieb erforderlichen Zulaufdruck gebracht. Für klassische Doppelkolonnenprozesse wird dabei typischerweise ein Speisedruckniveau im Bereich von etwa 5--6\,bar eingesetzt \cite{linde_engineering_air_2019}. Nach der Luftkompression erfolgt die Luftreinigung über Molekularsieb-Adsorber (\gls{TSA}/\gls{PSA}), wobei die Regeneration häufig mit Reststickstoff aus dem Kolonnenblock durchgeführt wird \cite{stefanie_janine_tesch_exergy-based_nodate, agrawal2000air}. Die Abkühlung bis nahe an die Verflüssigungstemperaturen erfolgt im Hauptwärmetauscher, der in modernen Anlagen als kompakter Mehrstromwärmetauscher ausgeführt ist.\\
Die vorgekühlte und gereinigte Luft wird zunächst in die Hochdruckkolonne (HPC) eingespeist, wo sie teilweise verflüssigt und grob in eine stickstoffreiche Kopfphase sowie eine sauerstoffangereicherte Flüssigkeit am Kolonnenboden getrennt wird. Diese Flüssigkeit wird anschließend in die Niederdruckkolonne (LPC) überführt, die bei einem niedrigeren Druckniveau betrieben wird \cite{hamayun_evaluation_2020, young_paper_2021, zhou_process_2012}. Dort erfolgt die Endtrennung in hochreinen Sauerstoff am Kolonnenboden und hochreinen Stickstoff am Kolonnenkopf \cite{smith_review_2001}. Für die industrielle Umsetzung werden insbesondere in der Niederdruckkolonne häufig strukturierte Packungen eingesetzt. Gegenüber Bodenkolonnen ermöglichen sie eine größere spezifische Austauschfläche bei gleichzeitig geringerem Druckverlust, wodurch sowohl die Trennleistung als auch der Energiebedarf vorteilhaft beeinflusst werden können \cite{heinz-wolfgang_haring_air_2007}.

Die Produktbereitstellung erfolgt abhängig vom geforderten Druckniveau und Aggregatzustand in unterschiedlichen Varianten. Während bei externer Kompression die gasförmigen Produkte nach Erwärmung im Hauptwärmetauscher mittels Gaskompressoren auf den Verbraucherdruck verdichtet werden, ist für Hochdruck-Sauerstoff in modernen Anlagen die interne Kompression mittels kryogener Flüssigkeitspumpe (pumped LOX) weit verbreitet. Dadurch können energieintensive und insbesondere bei Sauerstoff sicherheitskritische Gaskompressoren vermieden werden. Zugleich verringert das kontinuierliche Abziehen und Verdampfen von flüssigem Sauerstoff die Gefahr einer Kohlenwasserstoffanreicherung im Kolonnensumpf \cite{tesch_comparative_2020, heinz-wolfgang_haring_air_2007}. Abhängig vom Produktportfolio kann der Doppelsäulenprozess zusätzlich um eine Argonsektion erweitert werden, bei der ein geeigneter Zwischenstrom aus dem Kolonnenblock in einer nachgeschalteten Argonkolonne weiter aufgetrennt wird \cite{young_detailed_2021}. Auf Grundlage dieser Standardkonfiguration werden in der aktuellen Literatur sowohl Varianten der internen Produktkompression als auch alternative Kopplungskonzepte zwischen HPC und LPC sowie erweiterte Kolonnenanordnungen untersucht, um Effizienz und Produktflexibilität weiter zu erhöhen \cite{van_der_ham_exergy_2010, van_der_ham_improving_2011}.

\begin{figure}[h!]
    \centering
    \includegraphics[width=0.85\textwidth]{bilder/Luftzerlegung_Allgemein.png}
    \caption[Schematischer Aufbau einer kryogenen Luftzerlegungsanlage]
    {Schematischer Aufbau einer kryogenen Luftzerlegungsanlage zur Stickstoff- und Sauerstoffproduktion mit interner Kompression.}
    \label{fig:luftzerlegung_intern}
\end{figure}

Ein zentrales Merkmal moderner Luftzerlegungsanlagen ist die integrale Wärmeintegration innerhalb der isolierten Coldbox. Da tiefe Prozesstemperaturen ohne externe Kälteversorgung nur durch regenerative Kälterückführung aufrechterhalten werden können, bildet der Hauptwärmetauscher als Mehrstromwärmetauscher das Herzstück dieses Konzepts. Er kühlt die gereinigte Luft im Gegenstrom mit kalten Produkt- und Restgasströmen bis nahe an den Verflüssigungspunkt ab \cite{stefanie_janine_tesch_exergy-based_nodate}. In der Prozessmodellierung wird der Hauptwärmetauscher häufig gemeinsam mit Subcoolern und dem Kolonnenblock betrachtet, um die hohen Integrationsgrade realer Anlagen abzubilden \cite{hamayun_evaluation_2020}. Die zur Aufrechterhaltung des Kälteniveaus erforderliche Kälte wird in modernen Anlagen zusätzlich häufig über Entspannungsprozesse (z.\,B. Turbinenexpander) bereitgestellt und über das Wärmetauschersystem in den Prozess rückgeführt \cite{hamayun_evaluation_2020}.

Die thermische Kopplung der Kolonnen erfolgt im Doppelsäulenprozess über den Kondensator-Reboiler. Dieser verbindet die Kondensation des HPC-Stickstoffs am Kopf mit der Verdampfung des LPC-Sauerstoffs im Sumpf, wodurch externe Heiz- oder Kühlleistungen weitgehend vermieden werden \cite{hamayun_evaluation_2020}. Zur Minimierung irreversibler Verluste ist hierbei insbesondere die Reduktion von Temperaturdifferenzen und Druckverlusten entscheidend \cite{yan_energy_2010}. Pinch-basierte Analysen verdeutlichen, dass insbesondere Phasenübergänge die Lage des Pinch-Punkts dominieren. Als Zielgröße für eine effiziente Wärmeintegration werden in Fallstudien minimale Temperaturabstände in der Größenordnung weniger Kelvin (z.\,B. etwa \SI{3}{\kelvin}) diskutiert \cite{yan_energy_2010}.

Über die klassische punktuelle Kopplung hinaus untersuchen aktuelle Forschungsarbeiten intensivere Integrationsformen. So wird gezeigt, dass eine Verteilung der Wärmeübertragung über eine größere Kolonnenlänge (``HI-stages'') die Entropieproduktion senken kann \cite{van_der_ham_exergy_2010}. Dies wird technisch durch eine räumliche Verschiebung der LPC entlang der HPC realisiert. In modernen Systembetrachtungen rückt zudem die Dynamik in den Fokus: Stark gekoppelte und nichtlineare Wärme- und Stoffübertragungsprozesse können die Lastwechselfähigkeit begrenzen. Die Aufrechterhaltung stabiler Gegenstromstrukturen bei schnellen Lastwechseln erfordert daher spezialisierte Regelungsstrategien für die Wärmeintegration \cite{cheng_flexible_2022}.

Trotz der hohen technischen Reife gilt die kryogene Luftzerlegung aufgrund der erforderlichen Verdichtungs- und Kühlleistungen als energieintensiv. Zur systematischen Identifikation von Optimierungspotenzialen werden in der Literatur daher verstärkt Exergieanalysen herangezogen. Diese zeigen konsistent, dass die Hauptursachen der Exergiedestruktion in der Luftverdichtung sowie der anschließenden Abkühlung und Verflüssigung liegen, während die Rektifikationssektion vergleichsweise geringere, aber dennoch signifikante Verluste verursacht \cite{cornelissen_exergy_1998, bucsa_exergetic_2022, hamayun_evaluation_2020}. Insbesondere die Wärmeabfuhr nach den Kompressionsstufen stellt eine kritische Schwachstelle dar; ein sehr großer Anteil der gesamten Exergievernichtung wird dabei auf die Nachkühler der Verdichtung zurückgeführt \cite{van_der_ham_exergy_2010}, was die Notwendigkeit einer verbesserten thermischen Integration auch außerhalb der Coldbox unterstreicht.

Innerhalb des kryogenen Systems bestimmen vor allem Temperaturdifferenzen und Druckverluste die thermodynamische Güte. Die Minimierung dieser Irreversibilitäten, insbesondere am kalten Ende des Hauptwärmetauschers, ist daher ein zentrales Ziel aktueller Optimierungsstrategien \cite{yan_energy_2010}. In der Rektifikationssektion hat sich der Einsatz strukturierter Packungen als Stand der Technik etabliert, da diese im Vergleich zu Bodenkolonnen eine hohe Trennleistung bei minimalem Druckabfall $\Delta p$ ermöglichen und so die erforderliche Verdichterarbeit reduzieren können \cite{yan_energy_2010, heinz-wolfgang_haring_air_2007}.

Die technologische Weiterentwicklung moderner ASUs ist somit primär durch eine konsequente Reduktion irreversibler Verluste in den Schlüsselkomponenten geprägt. Dies umfasst neben hocheffizienten Expansionsmaschinen und optimierten Wärmetauschernetzwerken auch neuartige Kolonnenkonzepte mit intensiverer Wärmeintegration, etwa durch die Verteilung von Kondensations- und Verdampfungsaufgaben über zusätzliche integrierte Stufen \cite{van_der_ham_improving_2011}. Zusammenfassend charakterisiert sich der aktuelle Stand der Technik durch ein hochgradig integriertes Gesamtsystem, dessen Effizienz maßgeblich von der exergetischen Abstimmung zwischen Druckniveaus, Temperaturprofilen und Austauschflächen abhängt \cite{hamayun_evaluation_2020, yan_energy_2010}.

\subsection{Abgrenzung und Anforderungen zur reinen Stickstoffbereitstellung}
\label{sec:grundlagen_stickstoff}

In der kryogenen Luftzerlegung ist die Standardauslegung historisch und technisch auf die gleichzeitige Gewinnung von Stickstoff und Sauerstoff ausgelegt. Die ausschließliche Bereitstellung von Stickstoff (sogenannte „Nitrogen-only“-Anlagen oder Stickstoffgeneratoren) erfordert dagegen zielgerichtet angepasste verfahrenstechnische Konzepte, die insbesondere auf Stickstoffreinheit, Stickstoffausbeute und Produktdruck optimiert werden.\\

Agrawal und Woodward zeigen, dass das konventionelle Doppelsäulensystem „relativ ineffizient“ wird, sobald ausschließlich pressurisierter Stickstoff als Produkt benötigt wird und kein Sauerstoff oder Argon als Koppelprodukt gefordert ist. Die Ursache liegt vor allem in der thermischen Kopplung der beiden Kolonnen: Der notwendige Rücklaufbedarf diktiert den Boil-up am Sumpf der LP-Kolonne, obwohl die Trennung im unteren Kolonnenteil bei niedrigen Drücken vergleichsweise leicht ist und somit mehr Dampf erzeugt wird, als für eine effiziente Stickstoffabtrennung erforderlich wäre. Zusätzlich führt das niedrige Druckniveau der LP-Kolonne dazu, dass auch viele andere Prozessströme auf niedrigem Druckniveau liegen, was bei gegebenen Apparateabmessungen höhere Druckverlustanteile verursacht und dadurch große Apparatequerschnitte notwendig macht \cite{agrawal_efficient_1991}.\\

In der industriellen Praxis wird zur reinen Stickstoffbereitstellung daher häufig ein Ein-Kolonnen-Verfahren eingesetzt, bei dem auf die zweite Destillationskolonne (LPC) verzichtet wird. Agrawal und Thorogood beschreiben hierfür den weit verbreiteten \emph{single distillation column waste-expander process}, der insbesondere für die Bereitstellung von Stickstoff im mittleren Druckbereich (typisch ca. 5{,}5 bis 9{,}6~bar(abs.)) eingesetzt wird. In diesem Konzept wird Stickstoff am Kolonnenkopf als Produkt abgezogen, während ein Teil des Kopfgases kondensiert und als Rücklauf (Reflux) zur Aufrechterhaltung der Trennleistung dient \cite{agrawal_production_1991}.\\

Ein wesentlicher Unterschied gegenüber Standard-ASUs liegt in der Behandlung des Sauerstoffs: In klassischen Doppelsäulenanlagen stellt Sauerstoff ein wertvolles Produkt dar, das am Sumpf der Niederdruckkolonne gewonnen und je nach Anwendungsfall weiter verdichtet bzw. verflüssigt wird \cite{tesch_comparative_2020}. In N$_2$-only-Anlagen hingegen wird Sauerstoff typischerweise nicht als Produkt ausgefahren, sondern verlässt den Prozess als O$_2$-angereicherter Reststrom. Dieser übernimmt gleichzeitig eine wichtige Funktion im Kältekonzept, da er häufig entspannt (Turboexpander) und zur Kältebereitstellung genutzt wird. Die energetische Optimierung von N$_2$-only-Anlagen hängt damit wesentlich von der effizienten Nutzung dieses O$_2$-reichen Reststroms sowie der Minimierung von Bypass- und Drosselverlusten ab \cite{agrawal_production_1991}.\\

Aus den Reinheitsanforderungen ergibt sich außerdem eine typische „einseitige“ Optimierung der Rektifikation. Wie in der Tieftemperaturtechnik beschrieben, wird die Rektifikation in der Praxis häufig so betrieben, dass vor allem eines der Produkte in höchster Reinheit bereitgestellt wird, während das zweite Produkt nicht zwingend in hoher Reinheit vorliegen muss \cite{hausen_linde_1985}. Um eine extrem geringe O$_2$-Kontamination am Kolonnenkopf zu erreichen, muss die Trennleistung im oberen Teil der Säule durch eine höhere Anzahl theoretischer Böden bzw. eine entsprechend geeignete Kolonneninnenausstattung gesteigert werden. Dies führt zu einem direkten Trade-off: Eine höhere Reinheit reduziert bei gleicher Stufenzahl die Ausbeute (Yield), da eine geringere Produktentnahme bzw. ein höherer Rücklauf erforderlich ist, um Sauerstoffspuren konsequent in den Sumpf zu drücken \cite{haering_airgases_2007}. Spezielle \emph{High-Purity}-Anlagen nutzen hierfür häufig modifizierte Ein-Kolonnen-Systeme mit integrierten Kondensator-Verdampfer-Einheiten sowie zusätzlichen Strippsektionen, um hohe Reinheiten bei akzeptabler Ausbeute zu erreichen \cite{haering_airgases_2007}.\\

Darüber hinaus ergeben sich Unterschiede in der Luftvorreinigung und im Umgang mit Spül- bzw. Regenerationsgasen: Während in Standardanlagen Stickstoffströme häufig als Inertgas für verschiedene Hilfsfunktionen (z.\,B. als Spülgas oder zur Regeneration von Adsorbern in der Luftvorreinigung) verfügbar sind, ist in N$_2$-only-Anlagen der hochreine Stickstoff primär als Produktstrom zu betrachten. Eine Nutzung des Produktstickstoffs als Regenerations- oder Spülgas für die Adsorber der Luftvorreinigung (CO$_2$- und H$_2$O-Abscheidung) ist daher in der Regel nachteilig, da sie unmittelbar die verfügbare Produktmenge reduziert und damit den Zielkonflikt zwischen Reinheit und Ausbeute verschärfen kann (anlagenabhängig).\\

Ein prozesstechnischer Vorteil der reinen Stickstoffgewinnung ist zudem der Umgang mit Argon. In Standard-ASUs erschwert Argon die gleichzeitige Gewinnung von hochreinem Sauerstoff und Stickstoff, da sein Siedepunkt zwischen beiden Hauptkomponenten liegt und dadurch eine zusätzliche Komplexität in der Trennaufgabe entsteht \cite{hausen_linde_1985}. Bei stickstofffokussierten Anlagen ist man dagegen häufig toleranter gegenüber dem Verbleib von Argon im O$_2$-reichen Restgasstrom, was das Design der Trennstufen im Vergleich zur Dreistoff-Trennung (N$_2$/O$_2$/Ar) vereinfachen kann.\\

Ein weiterer Abgrenzungspunkt betrifft die Druckbereitstellung des Stickstoffprodukts: N$_2$-only-Anlagen werden häufig so ausgelegt, dass Stickstoff bei geringen bis mittleren Drücken direkt auf dem benötigten Produktdruckniveau aus der Coldbox bereitgestellt wird \cite{agrawal_production_1991}. Gleichzeitig zeigen sie auf, dass die direkte Bereitstellung sehr hoher Drücke aus der Coldbox mit einer deutlichen energetischen Einbuße (Energy Penalty) verbunden sein kann; in solchen Fällen kann es energetisch vorteilhafter sein, Stickstoff bei niedrigerem Druck zu erzeugen und anschließend extern zu komprimieren \cite{agrawal_production_1991}.\\

Darüber hinaus ist festzuhalten, dass die technischen Weiterentwicklungen, die in der Literatur primär für moderne Standard-Luftzerlegungsanlagen beschrieben und im Kapitel \ref{sec:grundlagen_doppelsäule} erläutert wurden, gleichermaßen den Stand der Technik für Stickstoffgeneratoren definieren. Auch wenn in der aktuellen Forschungsliteratur vergleichsweise wenige Publikationen existieren, die sich ausschließlich der dedizierten Weiterentwicklung von Stickstoff-optimierten Anlagen widmen, profitieren diese unmittelbar von den Fortschritten im allgemeinen ASU-Design (z.\,B. durch strukturierte Packungen oder optimierte Wärmeübertrager).

\begin{figure}[h!]
    \centering
    \includegraphics[width=0.85\textwidth]{bilder/Luftzerlegungsanlage_N2_Allgemein.png}
    \caption[Schematischer Aufbau einer stickstoffgenerierenden Luftzerlegungsanlage]
    {Schematischer Aufbau einer stickstoffgenerierenden Variante mit externer Kompression.}
    \label{fig:luftzerlegung_extern_n2}
\end{figure}

\subsection{Einsäulenprozess zur Stickstoffbereitstellung}
\label{sec:grundlagen_einsäulen}

Einkolonnenprozesse stellen eine vereinfachte Alternative zur klassischen 
Doppelkolonnentechnologie der kryogenen Luftzerlegung dar. Während sie in der 
Literatur traditionell aufgrund der geringen Nebenproduktgewinnung und 
vergleichsweise hoher Exergieverluste als energetisch unterlegen beschrieben 
wurden, ermöglichen moderne Weiterentwicklungen inzwischen eine deutlich höhere 
Effizienz und Produktreinheit. Besonders für die Stickstofferzeugung sind 
Einkolonnenmodelle attraktiv, da sie im Vergleich zur Doppelkolonne eine 
wesentlich geringere Anlagenkomplexität sowie einen reduzierten Energie- und 
Investitionsbedarf aufweisen und sich dadurch für kompakte und robuste 
Prozesskonzepte eignen.\\

Speziell stickstoffgenerierende Einkolonnenprozesse haben seit den 
1990er-Jahren wieder verstärkt an Bedeutung gewonnen. In diesem Zeitraum wurden 
zahlreiche Patente eingereicht und Prozessvarianten weiterentwickelt, die sich 
alle auf das grundlegende Konzept des Waste-Expander-Single-Column-
Prozesses zurückführen lassen. Dieses Verfahren, das aus vereinfachten 
Varianten des Claude-Prozesses hervorgegangen ist, gilt bis heute als 
Standardlayout für kompakte Stickstoffanlagen bei kleiner Kapazität (z.\,B. APSA von Linde \cite{linde_handbook_2011} oder JN Serie von TNSC \cite{masahiro_history_nodate}) und bildet 
die technologische Grundlage für viele der modernen Verbesserungsansätze. \\

Das grundlegende Einkolonnenmodell zur Stickstofferzeugung besteht aus einer einzelnen, auf mittlerem Druck (5{,}5--9{,}6~bar(abs.) \cite{agrawal_production_1991}) betriebenen 
Rektifikationskolonne, in welche die vorgekühlte und getrocknete Luft in den unteren 
Kolonnenteil eingeleitet wird. Dort erfolgt die Trennung in einen 
stickstoffreichen Kopfstrom und eine sauerstoffreiche Flüssigphase am 
Kolonnenboden. Der Stickstoff wird im Kopfkondensator teilweise verflüssigt und 
liefert so den notwendigen Reflux für die Rektifikation. Der 
sauerstoffreiche Sumpfstrom wird im Bodenverdampfer verdampft und anschließend 
über einen Turboexpander entspannt, der die für die Kältebilanz des Prozesses 
erforderliche Expansionsarbeit bereitstellt. Dieses einfache Schema bildet die 
Basis für sämtliche später entwickelten Prozessverbesserungen.\\

Als eine der ersten systematischen Verbesserungen des klassischen Einkolonnenmodells schlagen Agrawal und Thorogood \cite{agrawal_production_1991} die Kombination aus Abgasrückführung und optimierter Expanderintegration vor. Durch die teilweise Rückführung des sauerstoffreichen Abgasstroms sowie die gezielte Nutzung eines Expanders zur Kälteerzeugung kann die Stickstoffausbeute erhöht und der spezifische Energiebedarf reduziert werden. Gleichzeitig verringern diese Maßnahmen die im Grundprozess auftretenden Exergieverluste. Allerdings erfordert die Rückführung zusätzliche Verdichterleistung, und die Prozessführung wird im Vergleich zum einfachen Waste-Expander-Modell komplexer, sodass der Vorteil hauptsächlich in Anwendungen mit höherem Effizienzanspruch zur Geltung kommt.

Ein weiterer bedeutender Verbesserungsansatz von Agrawal und Thorogood \cite{agrawal_production_1991} betrifft die Erweiterung des Strippingbereichs und die Optimierung des Reboilers mit einem vorgeschlagenen Dual-Reboiler-Konzept.
Der erweiterte Strippingbereich ermöglicht es, dass aufsteigende Gasströme die sauerstoffreiche Flüssigphase im Kolonnenboden zusätzlich vom noch enthaltenen Stickstoff befreien. Dadurch sinken die Stickstoffverluste im Sumpf, was die Trennschärfe insbesondere im unteren Kolonnenabschnitt deutlich verbessert.
Beim vorgeschlagenen Dual-Reboiler wird der Kolonnenboden über zwei Reboiler auf unterschiedlichen Druckniveaus beheizt. Dadurch verringern sich die Temperaturdifferenzen im Wärmeaustausch, was die irreversiblen Verluste senkt und die Trennleistung im unteren Kolonnenbereich verbessert. Apparativ nähert sich dieses Konzept bereits einer Doppelkolonne an, da zwei getrennte Verdampferzonen genutzt werden, ohne jedoch deren vollständige Struktur zu benötigen.

Der Einsatz integrierter Wärmepumpen zur Effizienzsteigerung von Rektifikationsprozessen ist seit vielen Jahrzehnten bekannt. Insbesondere im Zusammenhang mit kryogenen Sauerstoffanlagen wurde die Verwendung von komprimiertem Kopfgas als Wärmepumpenfluid bereits mehrfach untersucht, da sich hierdurch die thermodynamischen Verluste am Reboiler deutlich reduzieren lassen.\\
Dabei wird der gasförmige Kopfstrom Stickstoff komprimiert, sodass seine Kondensation bei höherer Temperatur erfolgen kann. Die dabei freigesetzte Wärme dient direkt zur Verdampfung der sauerstoffreichen Flüssigphase am Kolonnenboden, während das kondensierte Stickstoffprodukt wieder als Reflux in die Kolonne zurückgeführt wird. Auf diese Weise schließt die Wärmepumpe den thermischen Kreisprozess nahezu vollständig und reduziert die irreversiblen Verluste, die im klassischen Waste-Expander-Modell auftreten.

Ein wesentlicher Nachteil integrierter Wärmepumpenkonzepte für stickstoffgenerierende Modelle ist das prinzipielle Risiko einer Produktkontamination, wenn ein Stickstoffstrom durch einen Kompressor geführt wird \cite{agrawal_production_1991}. Spätere Entwicklungen in der Verdichter- und Anlagentechnik zielen darauf ab, dieses Kontaminationsrisiko zu reduzieren; prinzipiell bleibt die Problematik jedoch bestehen.

Zhou et al. \cite{zhou_process_2012} untersuchen mehrere Varianten eines integrierten Wärmepumpenprozesses, speziell für die Einkolonnenmodelle, auch zur Stickstoffgenerierung. Sie zeigen, dass Varianten mit komprimiertem Stickstoffkopfgas die höchste Energieeffizienz und Stickstoffausbeute erreichen, jedoch aufgrund möglicher Produktkontamination nur eingeschränkt für hochreine Anwendungen geeignet sind. Alternativ untersuchte Konfigurationen, bei denen interne Prozessströme oder Luft anstelle des Produktstroms komprimiert werden, sind etwas weniger effizient, vermeiden jedoch die Reinheitsproblematik und sind daher praktischer umsetzbar.\\

Im Gegensatz zu den in der Forschung beschriebenen autarken Einkolonnenprozessen arbeiten in der Praxis kleine kommerzielle Stickstoffgeneratoren häufig weitgehend autark hinsichtlich der Kältebereitstellung (z.\,B. PRISM von Air Products \cite{airproducts_prism_2013} oder TCN-BE und CRYOSS von Linde \cite{linde_handbook_2011, Linde2003CryossEcovar}). Aufgrund der geringen Anlagengröße und der damit verbundenen begrenzten internen Kälteerzeugung wird ein Teil der benötigten Kälte über eine geregelte Einspritzung von Flüssigstickstoff aus einem externen Tank bereitgestellt. Dieses Konzept stabilisiert die Refluxerzeugung, vereinfacht die Regelung und senkt die Investitionskosten, ohne dass eine zweite Kolonne oder eine interne Wärmepumpe erforderlich ist.


\subsection{Doppelsäulenprozess zur Stickstoffbereitstellung}
\label{sec:grundlagen_doppelsäule_N2}

Während Einkolonnenprozesse eine kompakte und apparativ vereinfachte Lösung zur Stickstofferzeugung darstellen, wird in der industriellen Praxis für größere Durchsätze und höhere Rückgewinnungsgrade häufig ein zweisäuliges Trennkonzept eingesetzt. Der wesentliche Vorteil der Doppelsäule liegt dabei in der deutlich höheren Stickstoffausbeute (Recovery/Yield) gegenüber klassischen Einkolonnen-\emph{waste-expander}-Konfigurationen. Agrawal und Thorogood geben für einen konventionellen Einkolonnen-\emph{waste-expander}-Prozess einen Basiswert von etwa \SI{52.8}{\percent} Stickstoffrückgewinnung an, während verbesserte, stärker integrierte Konzepte Werte bis etwa \SI{80}{\percent} erreichen können \cite{agrawal_production_1991}. Für großtechnische Stickstoffgeneratoren sind in der Literatur zudem Rückgewinnungsgrade deutlich oberhalb von \SI{80}{\percent} beschrieben, was die Relevanz zweisäuliger Prozessvarianten insbesondere bei begrenzter oder teurer Luftbereitstellung unterstreicht \cite{haering_airgases_2007}.\\

Die höhere Ausbeute wird jedoch durch eine erhöhte Anlagenkomplexität erkauft. Im Vergleich zu Einkolonnenprozessen erfordert die Doppelsäule zusätzliche Kolonnenteile, Wärmeübertragerkopplungen und meist eine umfangreichere interne Verschaltung. Dies führt zu höheren Investitionskosten (CAPEX), kann jedoch bei großtechnischen Anlagen energetische Vorteile im Betrieb (OPEX) bringen, da bei gleicher Produktmenge weniger Zuluft verdichtet und abgekühlt werden muss. Agrawal und Woodward betonen in diesem Zusammenhang, dass konventionelle Doppelsäulenlayouts aus der Standard-ASU (O$_2$+N$_2$) für den Fall, dass ausschließlich pressurisierter Stickstoff benötigt wird, relativ ineffizient werden können. Ursache ist insbesondere die thermische Kopplung der Kolonnen, die Rücklauf- und Boil-up-Anforderungen erzwingt, welche für die reine Stickstoffproduktion nicht optimal sind \cite{agrawal_efficient_1991}. Daraus ergibt sich die Notwendigkeit spezifischer Doppelsäulenvarianten, die gezielt auf Stickstoff als einziges Produkt ausgelegt sind.\\

Verfahrenstechnisch entspricht der Stickstoffgenerator mit Doppelsäule grundsätzlich einer modifizierten Auslegung des klassischen HPC/LPC-Prinzips. Die aufbereinigte und abgekühlte Luft wird zunächst in die Hochdruck- bzw. Druckkolonne eingespeist, wo eine Vortrennung in einen stickstoffreichen Kopfstrom und eine sauerstoffangereicherte Flüssigkeit am Kolonnenboden erfolgt. Die sauerstoffreichere Flüssigkeit wird anschließend in die Niederdruckkolonne überführt, wo sie weiter rektifiziert wird. Die thermische Kopplung der beiden Kolonnen erfolgt über Kondensator/Verdampfer-Einheiten, wodurch Reflux und Boil-up ohne externe Heiz- bzw. Kälteleistungen bereitgestellt werden können \cite{haering_airgases_2007, agrawal_efficient_1991}. Im Unterschied zur Standard-ASU ist der Prozess dabei nicht auf einen hochreinen Sauerstoffstrom als Produkt ausgelegt: Der Sauerstoff wird vielmehr als O$_2$-reicher Reststrom geführt, der in geeigneten Konfigurationen auch zur Kältebereitstellung bzw. zur Energierückgewinnung genutzt werden kann \cite{agrawal_efficient_1991}.\\

Ein zentrales Merkmal vieler zweisäuliger Stickstoffgeneratoren ist die flexible Bereitstellung von Druckstickstoff. In energieoptimierten Konfigurationen wird ein signifikanter Anteil des Stickstoffprodukts direkt aus der Hochdruck- bzw. Druckkolonne entnommen, während der verbleibende Anteil aus der Niederdruckkolonne stammt. Häring beschreibt hierfür einen \emph{energy-optimized two-column nitrogen generator}, bei dem ein zusätzlicher Kondensator/Verdampfer am Kopf der Niederdruckkolonne eingesetzt wird, um dort Reflux zu erzeugen und damit den Bedarf an Rücklauf aus der Druckkolonne zu reduzieren. Dadurch kann ein größerer Anteil des am Kopf der Druckkolonne kondensierten Stickstoffs als Produkt ausgefahren werden; im beschriebenen Beispiel werden etwa \SI{48}{\percent} des Stickstoffprodukts aus der Druckkolonne bereitgestellt \cite{haering_airgases_2007}. Das Restgas verlässt die Anlage in diesem Konzept als O$_2$-reicher Abgasstrom, der nur noch einen relativ geringen Stickstoffanteil enthält (im Beispiel ca. \SI{25}{\percent} N$_2$), was die hohe Rückgewinnung des Stickstoffs zusätzlich verdeutlicht \cite{haering_airgases_2007}. Die Niederdruckkolonne übernimmt in solchen Konfigurationen damit teilweise die Funktion einer Wasch- bzw. Reinigungsstufe, während die Druckkolonne den Druckstickstoff bereitstellt.\\

Auch Agrawal und Woodward diskutieren zweisäulige Stickstoffgeneratoren, die im Vergleich zur klassischen Doppelsäule (für O$_2$+N$_2$) stärker auf die Stickstoffrückgewinnung und die Bereitstellung von Druckstickstoff ausgelegt sind. In einem beschriebenen Schema wird beispielsweise ein Anteil des Stickstoffs als Hochdruckprodukt direkt aus der Hochdruckkolonne entnommen, während Sauerstoff nicht als Produkt, sondern als O$_2$-reicher Reststrom betrachtet wird \cite{agrawal_efficient_1991}. Weiterhin zeigen sie, dass die Anhebung des Druckniveaus der Niederdruckkolonne (z.\,B. in den Bereich 2--4~bar) und die damit verbundene Verschiebung der Druckniveaus insgesamt zu vorteilhaften Betriebsbedingungen führen kann, da Druckverluste relativ reduziert werden und sich die Wärmeintegration über Kondensator/Verdampfer-Einheiten besser an den Stickstoffbetrieb anpassen lässt \cite{agrawal_efficient_1991}.\\

Neben diesen konzeptionellen Anpassungen existieren in der Literatur verschiedene Weiterentwicklungen, die sich insbesondere durch zusätzliche Kondensations- und Verdampfungsstufen innerhalb der zweisäuligen Anordnung auszeichnen. Agrawal und Woodward beschreiben hierzu Konfigurationen mit mehreren \emph{vaporizer/condensers} in der Bodensektion der Niederdruckkolonne, bei denen entweder Stickstoff oder Luft als kondensierendes Medium genutzt wird. Ziel dieser Konzepte ist es, die Temperaturdifferenzen im Wärmeaustausch zu reduzieren und damit die irreversiblen Verluste zu senken, was sich in einer verbesserten exergetischen Effizienz niederschlagen kann \cite{agrawal_efficient_1991}. In ähnlicher Richtung zeigen Agrawal und Thorogood, dass die Erweiterung des Grundprozesses um zusätzliche Kolonnensektionen und eine optimierte Reboiler-Verschaltung (z.\,B. Dual-Reboiler-Konzept in zweisäuliger Anordnung) die Stickstoffrückgewinnung gegenüber dem Basiskonzept deutlich steigern kann \cite{agrawal_production_1991}.\\

Zusammenfassend stellt der Doppelsäulenprozess für N$_2$-only-Anwendungen eine technisch anspruchsvollere, jedoch hinsichtlich Stickstoffrückgewinnung und energetischer Performance oft vorteilhafte Alternative zum Einkolonnenprozess dar. Insbesondere bei großen Produktmengen oder bei kostenintensiver Luftbereitstellung kann die höhere Ausbeute die zusätzlichen Investitionskosten rechtfertigen. Moderne zweisäulige Stickstoffgeneratoren kombinieren dabei die klassische thermische Kopplung mit gezielten Erweiterungen, wie zusätzlichen Kondensator/Verdampfer-Einheiten oder mehreren integrierten Wärmeaustauschstufen, um Druckstickstoff effizient und mit hoher Rückgewinnung bereitzustellen \cite{haering_airgases_2007, agrawal_efficient_1991, agrawal_production_1991}.




