%Dokumentklasse und Spracheinstellung
%für elektronische Veröffentlichung
\documentclass[12pt,oneside,paper=A4,DIV=15,BCOR=0mm,abstract=true,headsepline,headings=normal,ngerman]{scrreprt}
%für Druck
%\documentclass[12pt,twoside,paper=A4,DIV=15,BCOR=12mm,abstract=true,headsepline,headings=normal,ngerman]{scrreprt}
\usepackage{babel}
\usepackage[utf8]{inputenc}

%Schriftart
\usepackage{libertine}
\usepackage{libertinust1math}
\usepackage[T1]{fontenc}

%Mathe, Symbole, EInheitendarstellung, Chemie
\usepackage{amsmath}
\usepackage{amsxtra}
\usepackage{eurosym}
\usepackage{siunitx}  
\sisetup{locale=DE}
\usepackage[version=4]{mhchem}

\usepackage[auto]{microtype}
\clubpenalty = 10000
\widowpenalty = 10000
\displaywidowpenalty = 10000


%Glossary
\usepackage[acronym,nogrouptoc]{glossaries}
\newglossary[lg]{latin}{lat}{latgls}{Lateinische Symbole}
\newglossary[gg]{greek}{grk}{grkgls}{Griechische Symbole}
\makeglossaries


%Einbindung von Bildern, Tabellen, pdf-Seiten, Quellcode
\usepackage{graphicx}
\usepackage{multirow,multicol,booktabs}
\usepackage{threeparttable}
\usepackage{longtable}
\usepackage{rotating}
\usepackage{booktabs}
\usepackage{ltablex}
\usepackage{subfig}
\captionsetup[subtable]{position=top}
\usepackage{pdfpages}
\usepackage{listings}
\usepackage{wrapfig}
\usepackage{pdflscape}   % echte Querformat-Seite
\usepackage{adjustbox}   % automatische Skalierung


%Darstellung von URL
\usepackage{url}
\urlstyle{same}



%Pakete für Kontrolle und Review
\usepackage{todonotes}
\usepackage{blindtext}

%Darstellung der Literaturangaben
\usepackage[
backend=biber,
style=iso-numeric,
citestyle=numeric-comp,
maxbibnames=2,
firstinits=true
]{biblatex}

\renewcommand*{\labelnamepunct}{\addcolon\addspace}

%Speicherort der Literaturangaben (*.bib Datei)
\addbibresource{references.bib} % Dateiname exakt wie importiert!
%\bibliography{literatur/references}

%Fussnoten
%Markierung in der Fußnote selbst weder hochgestellt noch kleiner gesetzt
%\deffootnote{1em}{1em}{\thefootnotemark\ }
%linksbündige Fußnotenmarkierungen
\deffootnote{1.5em}{1em}{%
	\makebox[1.5em][l]{\thefootnotemark}%
}

%Fussnoten nicht umbrechen
\interfootnotelinepenalty=10000

%Gestaltung der Bildunterschrift und Tabellenüberschirften sowie Titelseitenangaben
\addtokomafont{caption}{\small}
\setkomafont{captionlabel}{\sffamily\bfseries}
\setkomafont{author}{\large}
\setkomafont{date}{\large}
\setkomafont{publishers}{\large}

\renewcaptionname{ngerman}{\figurename}{Abb.}
\renewcaptionname{ngerman}{\tablename}{Tab.} 

%Tabellenumgebungen mit Schriftgröße 10 und 7
\newenvironment{tabular10}{%
	\fontsize{10}{12}\selectfont\tabular
}{%
	\endtabular
}

\newenvironment{tabular7}{%
	\fontsize{7}{12}\selectfont\tabular
}{%
	\endtabular
}


%Verweise und Refernezen, pdf-Eisntellungen
%Angaben aktualisieren!
\usepackage[
pdftitle={Masterthesis Feline Ehresmann},
pdfsubject={},
pdfauthor={Feline Ehresmann},
pdfkeywords={},  
%Links nicht einrahmen
hidelinks
]{hyperref}
\usepackage[german]{cleveref}

