\chapter{Einleitung}
\label{ch:einleitung}

Die Geschwindigkeit \gls{v} hängt vom Winkel \gls{alpha} ab. 
Die \Gls{tub} ist eine technische Universität.

Denke an Abkürzungen

Relevanz der Themas

Relevanz der Luftzerlegung
Exergieanalyse statt Energieanalyse als relevanz

Relevanz wie viel mehr O2 und N2 und Argon in der Welt benötigt werden?!!

Motivation

Ziel der Arbeit

Problemstellung / ChalForschungslücke mit rein:

Während die klassische kryogene Luftzerlegung primär auf die Sauerstoffgewinnung optimiert ist, stellt die grüne Ammoniaksynthese (Power-to-Ammonia) neue Anforderungen an die Stickstoffbereitstellung. Bestehende Studien konzentrieren sich häufig auf Standard-ASUs oder basieren auf veralteten ökonomischen Indizes. Es besteht daher Bedarf an einer dedizierten Untersuchung, die spezifisch Einkolonnen- und Doppelkolonnen-Konzepte unter der Prämisse maximaler Stickstoffausbeute und -reinheit energetisch und wirtschaftlich gegenüberstellt. Insbesondere die exergetische Effizienz dieser reinen Stickstoffgeneratoren wurde in diesem Kontext bisher nur unzureichend vergleichend betrachtet.
Auch dass keine Simulationen zu Stickstoffgeneratoren sonden nur groben Aufbau...


Methode und aufbau/Gliederung??

Unterkapitel??
